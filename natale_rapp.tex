\documentclass[a4paper]{article}

\usepackage[english]{babel}
\usepackage[utf8]{inputenc}
\usepackage{amsmath}
\usepackage{amssymb}
\usepackage{graphicx}
\usepackage{ntheorem}
\usepackage[colorinlistoftodos]{todonotes}

\theoremstyle{break}
\newtheorem{ex}{{ \Large Esercizio} }
\theoremstyle{plain}
\newtheorem{sol}{Soluzione}[ex]

\title{Rappresentazioni \\ { \textit soluzioni agli esercizi per natale }}

\author{Second'anno di matematica, SNS}

\date{\today}

\begin{document}
\maketitle

\section*{Soluzioni agli esercizi}

\begin{ex} 
Siano $k$, e $n$ due interi positivi. Per ogni $\sigma \in S_k$ denotate con $\omega(\sigma)$ il numero di orbite di $\sigma$ su $\{1, \ldots, k\}$. Dimostrate la formula:
$$ \frac{1}{k!} \sum_{\sigma \in S_k} n^{\omega(\sigma)} = \binom{n+k-1}{k} $$

\begin{sol} 
	
\end{sol}

\begin{sol}

\end{sol}


\end{ex}

\begin{ex}
Calcolate la scomposizione in fattori irriducibili dei prodotti di tutte le possibili coppie di rappresentazioni irriducibili di $S_4$.

\begin{sol}

\end{sol}

\begin{sol}

\end{sol}


\end{ex}

\begin{ex}

\begin{itemize}
\item[(a)]  Sia $\rho: G \to \mathrm{GL}(V) $ una rappresentazione irriducibile di un gruppo $G$. Dimostrate che l’immagine del centro di G è contenuta nel sottogruppo dei multipli dell’identità.

\item[(b)] Dimostrate che ogni sottogruppo finito di $\mathbb{C}^*$ è ciclico

\item[(c)] Se un gruppo finito ha una rappresentazione fedele irriducibile, allora il suo centro è ciclico. Nota: una rappresentazione $\rho$ di $G$ è fedele se $\rho: G \to \mathrm{GL}(V_{\rho}) $ è iniettivo.

\end{itemize}
\begin{sol}

\end{sol}

\begin{sol}

\end{sol}


\end{ex}

\begin{ex}
Se $\rho$ è una rappresentazione irriducibile di $S_5$ di grado 5 e $s \in S_5$ è un 5-ciclo, fate vedere che $\rho(s)$ ha come autovalori tutte e sole le radici quinte dell’unità.

\begin{sol}

\end{sol}

\begin{sol}

\end{sol}


\end{ex}

\begin{ex}
Trovare la tavola dei caratteri di $A_4$.

\begin{sol}
	Do per noto che le classi di coniugio di $A_4$ siano rappresentate da $\text{id}$, $(1 2)(3 4)$, $(1 2 3)$, $(1 3 2)$ (si fa a conti ricordandosi che le classi di coniugio di $A_n$ sono o quelle di $S_n$ oppure quelle di $S_n$ spezzate a metà). \\
	$A_4$ ha $12$ elementi e $4$ classi di coniugio. Inoltre so che non è abeliano (quindi ha almeno una rappresentazione di dimensione $\ge 2$) e ammette sicuramente la rappresentazione banale. Facendo i casi si vede che l'unico modo di fare $12$ sommando quattro quadrati con questi constraint è $12 = 3^2 + 3 \cdot 1^2$. Andiamo quindi a cercare tre omomorfismi di $A_4$ in $\mathbb{C}^{*}$ per poi completare la tabella per ortogonalità. \\
	Notiamo che $A_4$ è generato dalla classe di coniugio di $(1 2 3)$. Infatti se conosco il valore di $\chi$ su questa classe so che $\chi (1 3 2) = \chi((1 2 3)) ^2$ e poi completo per omomorfismo. Siccome $(1 2 3)$ ha ordine tre e stiamo cercando rappresentazioni di grado $1$, si ha che $\chi (1 2 3)$ è un radice terza dell'unità. Ci sono quindi al più tre possibilità per un tale omomorfismo. Ciò significa che li abbiamo trovati tutti. \\
	Completando per ortogonalità si ricava:
	\begin{center} \begin{tabular}{lcccc}
	numero di elementi & 1   & 3   & 4   & 4   \\
	classi di conj. & $\text{id}$ & $(1 2)(3 4)$ & $(1 2 3)$ & $(1 3 2)$ \\ \hline
	$\chi_\text{id}$ & 1    & 1    & 1    & 1    \\
	$\chi_\text{a}$  & 1    & 1    & $\zeta$ & $\zeta^2$ \\
	$\chi_\text{b}$  & 1    & 1    & $\zeta^2$ & $\zeta$ \\
	$\chi_\text{g}$  & 3    & -1   & 0    & 0    \\
	\end{tabular} \end{center} \vskip 0.5cm
\end{sol}

\begin{sol}

\end{sol}


\end{ex}

\begin{ex}
Trovare la tavola dei caratteri di $D_4, D_5$.

\begin{sol}

\end{sol}

\begin{sol}

\end{sol}


\end{ex}

\begin{ex}
Sia $T$ un tetraedro di centro nell'origine, e sia $G \subseteq O_3$ il gruppo delle trasformazioni ortogonali che portano $T$ in se stesso. Numerando in qualche modo i vertici di $T$ da $1$ a $4$, otteniamo un’azione di $G$ su $\{1, 2, 3, 4\}$.
\begin{itemize}
\item[(a)] Fate vedere che questo dà un’identificazione di $G$ con $S_4$.

\item[(b)] Fate vedere che il sottogruppo di $G$ delle matrici con determinante positivo corrisponde ad $A_4$.

\item[(c)] Scomponete la rappresentazione per permutazioni corrispondente agli spigoli del tetraedro come rappresentazione di $A_4$.
\end{itemize}

\begin{sol}

\end{sol}

\begin{sol}

\end{sol}


\end{ex}

\begin{ex}


\begin{sol}

\end{sol}

\begin{sol}

\end{sol}


\end{ex}

\begin{ex}


\begin{sol}

\end{sol}

\begin{sol}

\end{sol}


\end{ex}

\newpage
\section*{Idee utili per gli esercizi}

\end{document}
