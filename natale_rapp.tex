\documentclass[a4paper]{article}

\usepackage[english]{babel}
\usepackage[utf8]{inputenc}
\usepackage{amsmath}
\usepackage{amssymb}
\usepackage{graphicx}
\usepackage{ntheorem}
\usepackage[colorinlistoftodos]{todonotes}

\theoremstyle{break}
\newtheorem{ex}{{ \Large Esercizio} }
\theoremstyle{plain}
\newtheorem{sol}{Soluzione}[ex]

\newcommand{\bbC}{\mathbb{C}}
\newcommand{\Irr}{\text{Irr }}

% Titolo
\title{Rappresentazioni \\ { \textit soluzioni agli esercizi per natale }}
\author{Second'anno di matematica, SNS}
\date{\today}

\begin{document}
\maketitle

\section*{Soluzioni agli esercizi}
% Esercizio 1
\begin{ex} 
Siano $k$, e $n$ due interi positivi. Per ogni $\sigma \in S_k$ denotate con $\omega(\sigma)$ il numero di orbite di $\sigma$ su $\{1, \ldots, k\}$. Dimostrate la formula:
$$ \frac{1}{k!} \sum_{\sigma \in S_k} n^{\omega(\sigma)} = \binom{n+k-1}{k} $$

\begin{sol} 
	
\end{sol}

\begin{sol}

\end{sol}


\end{ex}

% Esercizio 2
\begin{ex}
Calcolate la scomposizione in fattori irriducibili dei prodotti di tutte le possibili coppie di rappresentazioni irriducibili di $S_4$.

\begin{sol}

\end{sol}

\begin{sol}

\end{sol}


\end{ex}

% Esercizio 3
\begin{ex}

\begin{itemize}
\item[(a)]  Sia $\rho: G \to \mathrm{GL}(V) $ una rappresentazione irriducibile di un gruppo $G$. Dimostrate che l’immagine del centro di G è contenuta nel sottogruppo dei multipli dell’identità.

\item[(b)] Dimostrate che ogni sottogruppo finito di $\mathbb{C}^*$ è ciclico

\item[(c)] Se un gruppo finito ha una rappresentazione fedele irriducibile, allora il suo centro è ciclico. Nota: una rappresentazione $\rho$ di $G$ è fedele se $\rho: G \to \mathrm{GL}(V_{\rho}) $ è iniettivo.

\end{itemize}
\begin{sol}

\end{sol}

\begin{sol}

\end{sol}


\end{ex}

% Esercizio 4
\begin{ex}
Se $\rho$ è una rappresentazione irriducibile di $S_5$ di grado 5 e $s \in S_5$ è un 5-ciclo, fate vedere che $\rho(s)$ ha come autovalori tutte e sole le radici quinte dell’unità.

\begin{sol}

\end{sol}

\begin{sol}

\end{sol}


\end{ex}

% Esercizio 5
\begin{ex}
Trovare la tavola dei caratteri di $A_4$.

\begin{sol}
	Do per noto che le classi di coniugio di $A_4$ siano rappresentate da $\text{id}$, $(1 2)(3 4)$, $(1 2 3)$, $(1 3 2)$ (si fa a conti ricordandosi che le classi di coniugio di $A_n$ sono o quelle di $S_n$ oppure quelle di $S_n$ spezzate a metà). \\
	$A_4$ ha $12$ elementi e $4$ classi di coniugio. Inoltre so che non è abeliano (quindi ha almeno una rappresentazione di dimensione $\ge 2$) e ammette sicuramente la rappresentazione banale. Facendo i casi si vede che l'unico modo di fare $12$ sommando quattro quadrati con questi constraint è $12 = 3^2 + 3 \cdot 1^2$. Andiamo quindi a cercare tre omomorfismi di $A_4$ in $\mathbb{C}^{*}$ per poi completare la tabella per ortogonalità. \\
	Notiamo che $A_4$ è generato dalla classe di coniugio di $(1 2 3)$. Infatti se conosco il valore di $\chi$ su questa classe so che $\chi (1 3 2) = \chi((1 2 3)) ^2$ e poi completo per omomorfismo. Siccome $(1 2 3)$ ha ordine tre e stiamo cercando rappresentazioni di grado $1$, si ha che $\chi (1 2 3)$ è un radice terza dell'unità. Ci sono quindi al più tre possibilità per un tale omomorfismo. Ciò significa che li abbiamo trovati tutti. \\
	Completando per ortogonalità si ricava:
	\begin{center} \begin{tabular}{lcccc}
	numero di elementi & 1   & 3   & 4   & 4   \\
	classi di conj. & $\text{id}$ & $(1 2)(3 4)$ & $(1 2 3)$ & $(1 3 2)$ \\ \hline
	$\chi_\text{id}$ & 1    & 1    & 1    & 1    \\
	$\chi_\text{a}$  & 1    & 1    & $\zeta$ & $\zeta^2$ \\
	$\chi_\text{b}$  & 1    & 1    & $\zeta^2$ & $\zeta$ \\
	$\chi_\text{g}$  & 3    & -1   & 0    & 0    \\
	\end{tabular} \end{center} \vskip 0.5cm
\end{sol}

\begin{sol}

\end{sol}


\end{ex}

% Esercizio 6
\begin{ex}
Trovare la tavola dei caratteri di $D_4, D_5$.

\begin{sol}
\subsection{Tabella dei caratteri di D4}
	Ordine di D4: 8 elementi. Generato da 2 elementi: $\rho, \sigma$ (rispettivamente rotazione e simmetria, con relazioni $\rho^4 = \sigma^2 = e$ e $\sigma \rho \sigma^{-1} = \rho^{-1}$). Ha 5 classi di coniugio: $\{e\}, \{\rho^2\}, \{\rho, \rho^3\}, \{\sigma, \sigma\rho^2\}, \{\sigma\rho, \sigma\rho^3\}$. Non è abeliano e ha almeno la rappresentazione banale, quindi si ha $8 = 2^2 + 4 \cdot 1^2$ è l'unico modo di scrivere $8$. Dobbiamo quindi trovare $4$ omomorfismi di D4 in $\bbC^{*}$ per poi ricavare per ortogonalità la rappresentazione di dimensione $2$. \\
	Notiamo che, siccome $\sigma$ ha ordine $2$, esso può essere mandato solo in $\pm 1$ (radici 2-esime dell'unità) e siccome $\rho$ e $\rho^-1$ stanno nella stessa classe di coniugio si ha $x = \chi(\rho)$ deve essere una radice quarta dell'unità che rispetti $x = x^3$ e quindi deve essere solo o $1$ o $-1$. Abbiamo quindi solo quattro possibili scelte per un possibile omomorfismo da D4 in $\bbC^{*}$, che sono quindi obbligate perché sappiamo che esistono $4$ caratteri di dimensione $1$ per D4 (ovvero omomorfismi). Quindi scrivendo questi nella tabella e completandola per ortonormalità si ha: \\
	\begin{center} \begin{tabular}{lccccc}
	numero elementi   & 1   & 1   & 2   & 2   & 2   \\
	classi di conj.   & $e$  & $\rho^2$ & $\rho, \rho^3$ & $\sigma, \sigma\rho^2$ & $\sigma\rho, \sigma\rho^3$ \\ \hline
	$\chi_\text{id}$  & 1   & 1   & 1   & 1   & 1   \\
	$\chi_\text{a}$   & 1   & 1   & 1   & -1  & -1  \\
	$\chi_\text{b}$   & 1   & 1   & -1  & 1   & -1  \\
	$\chi_\text{ab}$  & 1   & 1   & -1  & -1  & 1   \\
	$\chi_\text{g}$   & 2   & -2  & 0   & 0   & 0   \\
	\end{tabular} \end{center} \vskip 0.5cm
\end{sol}

\begin{sol}

\end{sol}


\end{ex}

% Esercizio 7
\begin{ex}
Sia $T$ un tetraedro di centro nell'origine, e sia $G \subseteq O_3$ il gruppo delle trasformazioni ortogonali che portano $T$ in se stesso. Numerando in qualche modo i vertici di $T$ da $1$ a $4$, otteniamo un’azione di $G$ su $\{1, 2, 3, 4\}$.
\begin{itemize}
\item[(a)] Fate vedere che questo dà un’identificazione di $G$ con $S_4$.

\item[(b)] Fate vedere che il sottogruppo di $G$ delle matrici con determinante positivo corrisponde ad $A_4$.

\item[(c)] Scomponete la rappresentazione per permutazioni corrispondente agli spigoli del tetraedro come rappresentazione di $A_4$.
\end{itemize}

\begin{sol}

\end{sol}

\begin{sol}

\end{sol}


\end{ex}

% Esercizio 8
\begin{ex}
Sia $G$ un gruppo finito che agisce su un insieme finito $I$ con almeno due elementi in modo non banale. Diciamo che l'azione è {\it doppiamente transitiva} se date due coppie $(i, i')$ e $(j, j')$ di elementi di $I$ con $i \neq i'$ e $j \neq j'$ esiste $s \in G$ con $s \cdot i = j$ e $s \cdot i' = j'$. \\
Sia $\rho_I$ la corrispondente rappresentazione per permutazioni e scriviamo $\rho_I = \mathbb{1} + \rho$, come al solito. Allora fate vedere che $\rho$ è irriducibile se e solo se l'azione di $G$ su $I$ è doppiamente transitiva. \\
Concludete che $S_n$ ha una rappresentazione irriducibile di grado $n-1$ per ogni $n \ge 2$ \\
Cenno di soluzione: Fate vedere che $\rho$ è irriducibile se e solo se $\langle \chi_{\rho_I} \mid \chi_{\rho_I} \rangle = 2$. Ma $\chi_{\rho_I}$ è reale, per cui $\langle \chi_{\rho_I} \mid \chi_{\rho_I} \rangle = \langle \chi_{\rho_I}^2 \mid 1 \rangle = \langle \chi_{\rho_{I\times I}} \mid 1 \rangle$ e $\langle \chi_{\rho_{I\times I}} \mid 1 \rangle$ è il numero di orbite di $G$ su $I \times I$.

\begin{sol}

\end{sol}

\begin{sol}

\end{sol}


\end{ex}

% Esercizio 9
\begin{ex}
Sia $G$ un gruppo finito. Dati due elementi $h, g \in G$ appartenenti a classi di coniugio distinte mostrare che $$ \sum_{\rho \in \Irr(G)} \chi_\rho(g) \overline{\chi_rho(h)} = 0 $$
dove $\Irr(G)$ denota l'insieme delle rappresentazioni irriducibili di $G$ a meno di isomorfismo. Cosa riuscite a dire sulla precedente somma se invece $h$ e $g$ sono elementi coniugati? \\
Cenno di soluzione: Considerate il carattere di $\bbC[G]$ come rappresentazione di $G \times G$

\begin{sol}

\end{sol}

\begin{sol}

\end{sol}


\end{ex}

\newpage
\section*{Idee utili per gli esercizi}

\end{document}
